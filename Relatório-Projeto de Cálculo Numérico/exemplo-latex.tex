% Modelo de projeto para disciplinas do Abel

\documentclass[a4paper,11pt]{article}
\usepackage{sbpo-template}
\usepackage[utf8]{inputenc}
\usepackage[brazil]{babel}
\usepackage{amsmath,amssymb}
\usepackage{url}

\usepackage[numbers,sort&compress]{natbib}
\usepackage{indentfirst}
\usepackage{mathpazo}
\usepackage{eulervm}

\title{Equação do Calor undimensional no tempo}

\begin{document}

\maketitle

\author{
  \name{Ana Flávia Sturaro Calegari}
  \name{Danyelle Horobinski}
  \name{Jéssica Gomes Furtado}
  \name{Lucas Gabriel Nadolny}
  \institute{UFPR - Disciplinas - Semestre}
}

\vspace{8mm}
% \begin{resumo}
%  Resumo aqui.

% \end{resumo}

% \bigskip
% \begin{palchaves}
% Primeira, Segunda, Terceira.

% \bigskip

% \end{palchaves}

%%%%%%%%%%%%%%%%%%%%
\section{Introdução}


\section{Equação do Calor}
A Equação do Calor representa a variação da temperatura em um determindo espaço, após decorrido um determinado intervalo de tempo. A mesma é formada a partir de equações difereciais parciais, compostas por duas variáveis independentes $x$ e $t$, uma variável dependente $u(x,t)$, além de uma constante $\alpha$ que mede a difusividade térmica. \\ \indent Sendo assim, a Equação do Calor dá-se por: 

%equação do calor
$$\dfrac{\partial u}{\partial t}= \alpha \dfrac{\partial^2 u}{\partial x^2}$$

\\ \indent Tal equação é aplicada a uma barra, formada de material condutor, de comprimento $L$. Dessa forma, $0 \leq x \leq L$ e $0 \leq t \leq tmax$, isto é, $t$ é um intervalo de tempo finito. Sendo assim, através da Equação do Calor expressa-se o equilíbrio  térmico, no qual a taxa de entrada de calor em uma parte dessa barra é equivalente a taxa de absorção nessa mesma parte. \\
\indent Para resolver tal equação, é necessário considerar as seguintes condições de contorno para $x=0$, $x=L$ e $t=0$:
$$u(0,t)=u_0$$ $$u(L,t)=u_L$$ $$u(x,0)=f_0(x)$$
\section{Métodos Numéricos}

\subsection{Discretização}
Para que seja possível resolver problemas diferenciais computacionalmente faz-se necessário identificar a região de interesse para a resolução do problema, pois, quando se trata de uma região contínua há uma infinidade de pontos envolvidos, e, sendo assim, não é possível estabelecer soluções numéricas para o problema. \\ \indent Dessa forma, a região de interesse é discretizada, ou seja, é substituída por um conjunto finito de pontos e a solução será obtida apenas nesses pontos. \\ \indent Portanto, com a discretização, o espaço (comprimento da barra) é dividido em interválos equidistantes de tamanho $\Delta x$, assim como o tempo, cujos intervalos são de tamanho $\Delta t$. Dessa forma, os pontos a serem considerados no problema são representados por uma malha. \\ \indent Os pontos são determinados em um intervalo de $0 \leq x \leq L$ tais que:

$$x_i = (i-1)\Delta x, \indent i=1, ... , N$$ 
\\ \indent Sendo N a quatidade total de pontos. Dessa forma, o passo (distância do intervalo) dá-se por:
$$\Delta x=\frac{L}{(N-1)}$$
\indent De maneira similar, também o tempo $t$ em um intervalo de $0 \leq t \leq tmax$ é dividido, tal que:
$$t_m=(m-1)\Delta t, \indent m=1, ... , M$$
\indent Onde, $M$ é o total de pontos e $\Delta t$ calcula-se por:
$$\Delta t=\frac{tmax}{(M-1)}$$

%imagem da malha
%perguntar sobre as representação dos intervalos

\subsection{Aproximação de derivada de primeira ordem por diferença}
Através do Polinômio de Taylor com Resto de Lagrange é possivel aproximar funções com certo nível de precisão. A aproximação é calculada da seguinte forma:

$$f(x)=f(x_0)+f'(x_0)(x-x_0)+\frac{f''(x_0)(x-x_0)^2}{2!}+\frac{f'''(x_0)^3}{3!}+ ... +\frac{f^{(n)}(x_0)(x-x_0)^n}{n!}+\frac{f^{(n+1)}(c)(x-x_0)^{(n+1)}}{(n+1)!}$$
\indent Considerando $g(x)=u(x,t_m)$, com $t_m$ fixo, e aplicando o Polinômio de Taylor com Resto de Lagrange ao redor de $x_0=x_i$, tem-se:

$$g(x)=g(x_i)+g'(x_i)(x-x_i)+\frac{g''(c)(x-x_i)^2}{2} \eqno(1)$$
\indent Sendo $h=\Delta x$ e $x=x_i+h$, tem-se:
$$g(x_i+h)=g(x_{i+1})=g(x_i)+g'(x_i)h+\frac{g''(c)h^2}{2} \eqno(2)$$
\indent Isolando $g'(x_i)$ em (1), obtêm-se:
$$g'(x_i)=\frac{g(x_{i+1})-g(x_i)}{h}-\frac{g''(c)h}{2} \eqno(3)$$
\indent Com $x_i \leq c \leq x_{x+1}$. Isolando o erro na equação (2), tem-se:
$$-\frac{g''(c)h}{2}=-g'(x_i)+\frac{g(x_{i+1})-g(x_i)}{h}$$
\indent Dessa forma, o erro depende de $h$. Sendo, $\Phi (h)=-\frac{g''(c)h}{2}$, obtêm-se:
$$g'(x_i)=\frac{g(x_{i+1})-g(x_i)}{h}+\Phi (h) \eqno (4)$$
\indent A equação (4) é denominada \textit{forward difference} por envolver $x_i$ e $x_{i+1}$. O erro pode ser controlado ao escolher $h$, quanto menor o $h$ menor será o erro, todavia, em relação a componente $g''(c)$ não há qualquer tipo de controle. \\ \indent Analogamente, subtituindo $x=x_i-h$ na equação (1), chega-se em:
%$$g(x_i-h)=g(x_{i-1})=g(x_i)-g'(x_i)h+\frac{g''(c)h^2}{2} \eqno(5)$$
%\indent Fazendo as manipulações, tem-se:
$$g'(x_i)=\frac{g(x_i)-g(x_{i-1})}{h}+\Phi (h) \eqno (5)$$
\indent A equação (5) é chamada de \textit{backward difference}, pois envolve $x_i$ e $x_{i-1}$. O erro é o mesmo tanto para a equação (3) quanto para a (4), porém, ainda é possivel chegar em uma aproximação cujo erro é menor.
%colocar a diferença central?
%se não colocar lembrar de tirar a última frase do último parágrafo
\subsubsection{Aproximação de derivada de segunda ordem por diferença}

Para aproximar derivadas de segunda ordem, é preciso da Fórmula de Taylor até terceira ordem com Resto de Lagrange aplicada em $g(x_{i+1})$ e $g(x_{i-1})$. Portanto, tem-se:
$$g(x_{i+1})=g(x_i)+g'(x_i)h+\frac{g''(x_i)h^2}{2}+\frac{g'''(x_i)h^3}{3!}+\frac{g^{iv}(c)h^4}{4!} \eqno(6)$$
\indent e
$$g(x_{i-1})=g(x_i)-g'(x_i)h+\frac{g''(x_i)h^2}{2}-\frac{g'''(x_i)h^3}{3!}+\frac{g^{iv}(c)h^4}{4!}  \eqno(7)$$
\indent Logo, somando (6) e (7), obtêm-se:

$$g(x_{i+1})+g(x_{i-1})=2g(x_i)+g''(x_i)h^2+\frac{g^{iv}(c)h^4}{12} \eqno (8)$$
\indent Isolando $g''(x_i)$ em (8), tem-se:
$$g''(x_i)=\frac{g(x_{i+1})-2g(x_i)+g(x_{i-1})}{h^2}-\frac{g^{iv}(c)h^2}{12} \eqno(9)$$ 
\indent Substituindo $-\frac{g^{iv}(c)h^2}{12}=\Phi (h^2)$ em (9), tem-se:
$$g''(x_i)=\frac{g(x_{i+1})-2g(x_i)+g(x_{i-1})}{h^2}+\Phi(h^2) \eqno(10)$$ 
\indent A equação (10) é chamada aproximação de diferença central.
\section{Métodos de Diferença Finita (MDF)}
%colocar alguma coisa
\subsection{Diferença Finita Progressiva}
Como a equação do calor dá-se por:
$$\dfrac{\partial u}{\partial t}= \alpha \dfrac{\partial^2 u}{\partial x^2} \eqno(11)$$
Será utilizada a equação \textit{forward difference} (4), para aproimar a deriada de primeira ordem, o primeiro membro da equação (11). Como a derivada em questão é em relação ao tempo, fazendo as manipulações, tem-se:
$$\dfrac{\partial u}{\partial t}=\frac{u(x_i,t_{m+1})-u(x_i,t_{m})}{\Delta t}+\Phi (\Delta t) \eqno (12)$$
\indent Para aproximar a derivada de segunda ordem, o segundo membro da equação, será utilizada a equação (10), logo, tem-se:
$$\dfrac{\partial^2 u}{\partial x^2}=\frac{u(x_{i+1},t_m)-2g(x_i,t_m)+g(x_{i-1},t_m)}{\Delta x^2}+\Phi(\Delta x^2) \eqno(13)$$ 
\indent Fazendo as respectivas substituições das equações (12) e (13) na equação (11), chega-se em:
$$\frac{u(x_i,t_{m+1})-u(x_i,t_{m})}{\Delta t}+\Phi (\Delta t)=\alpha (\frac{u(x_{i+1},t_m)-2u(x_i,t_m)+u(x_{i-1},t_m)}{\Delta x^2}+\Phi(\Delta x^2)) \eqno (14)$$
\indent Isolando $u(x_i,t_{m+1})$ e eliminando o erro na equação (14), obtém-se:
$$u(x_i,t_{m+1})=u(x_i,t_{m})+\frac{\alpha \Delta t}{\Delta x^2}[u(x_{i+1},t_m)-2u(x_i,t_m)+u(x_{i-1},t_m)] \eqno(15)$$
\indent Substituindo $r=\frac{\alpha \Delta t}{\Delta x^2}$, fazendo a distrisibutiva e evidenciando $u(x_i,t_{m})$ na equação (15), obtêm-se:
$$u(x_i,t_{m+1})=u(x_i,t_m)(1-2r)+ru(x_{i+1},t-m)+ru(x_{i-1},t_m) \eqno (16)$$
\indent Dessa forma, pela equação (15) obtêm-se uma aproximação para e Equação do Calor. Todavia, essa aproximação mostra-se instável em determinadas situações, para que esse método retorne soluções estáveis, é preciso que:

$$r=\frac{\alpha \Delta t}{\Delta x^2}<\frac{1}{2} \eqno (17)$$

\indent Como $\Delta t=\frac{tmax}{(M-1)}$ e $\Delta x=\frac{L}{(N-1)}$, substituindo em (17), tem-se:

$$r=\alpha \frac{tmax (N-1)^2}{L^2 (M-1)}< \frac{1}{2}$$
\indent Como $L$ é o comprimento da barra, $tmax$ é o tempo em que será medido a variação de temperatura, $\alpha$ é a constante que mede a difusividade térmica, e o vetor do espaço com suas partições é fornecido, logo, essas componentes não podem ser controladas. Sendo assim, para que o método obtenha soluções estáveis é preciso controlar $(M-1)$ (total de intervalos de tempo), ou seja, $M$ precisa ser suficientemente "grande" para que sempre $r< \frac{1}{2}$.





%colocar imagem da malha
\section{Metodologia e Análise}

\section{Considerações finais}

%%%%%%%%%%%%%%%%%%%%

\bibliographystyle{sbpo} %% Não precisa mexer
\bibliography{exemplo-latex} %% Não precisa mexer

\end{document}

